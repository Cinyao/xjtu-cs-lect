\def\lecture{22}
\clearpage \noindent\begin{tabularx}{\linewidth}{|X|}
\hline \vskip -2mm
{\sf 编译原理} \hfill November 18, 2011 \\
{\centering \sf \large Lecture \lecture:
符号表 \\ }
\textsl{Lecturer: 冯博琴 \hfill Scriber: 戴唯思}\\ \hline
\end{tabularx}
\setcounter{section}{0}
\renewcommand{\thepage}{\lecture -\arabic{page}}

\section{符号表}

    组织方案: 线性表, 折半查找, 哈希表. 此外还要解决冲突问题, 因此需要引入探查机制: (内部解决) 线性探查法(开放探查法), 二次探查法, 平方探查法, (外部解决) 链表法, \ldots

    \paragraph{关于哈希表的讨论}

        总时间$=$计算时间$I$+查找长度. 因此尽量减小查找长度.
