\def\lecture{11}
\clearpage \noindent\begin{tabularx}{\linewidth}{|X|}
\hline \vskip -2mm
{\sf 编译原理} \hfill October 9, 2011 \\
{\centering \sf \large Lecture \lecture:
自下而上的语法分析 \\ }
\textsl{Lecturer: 冯博琴 \hfill Scriber: 戴唯思}\\ \hline
\end{tabularx}
\setcounter{section}{0}
\renewcommand{\thepage}{\lecture -\arabic{page}}

\newcommand\eqdot{=\hskip -.8em\cdot\ }

\section{自下而上的语法分析}

    \subsection{移进--- 归约}

        动作:
        \begin{description}
            \item[移进] 下一输入符号移进栈顶
            \item[归约] 把句柄按产生式的左部进行归约
            \item[接受] 分析程序报告成功
            \item[出错] 发现了一个语法错, 调用出错处理程序
        \end{description}

        被归约的串总是出现在栈顶.

    \subsection{直观算符优先分析法}

        任二个相继出现的终结符$a$与$b$(可能中间有$V_N$)的优先级可能有以下优先关系: $a\lessdot b$, $a\eqdot b$, $a\gtrdot b$. 采用\textbf{优先级表}\footnote{注: 我想到了中缀表达式转后缀表达式的算法\ldots}\footnote{明神指出, 严蔚敏《数据结构》里有这个例子, 一模一样}!

    \subsection{算符优先分析}

        \paragraph{算符优先文法}

            如果一个文法的任何产生式右部都不含两个相继(并列)的非终结符, 即不含有$\cdots QR\cdots$形式的产生式右部则我们称该文法为\textsf{算符文法}.

            % TODO

    \subsection{LR分析法}
