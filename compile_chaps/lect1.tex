\clearpage \noindent\begin{tabularx}{\linewidth}{|X|}
\hline \vskip -2mm
{\sf 编译原理} \hfill August 31, 2011 \\
{\centering \sf \large Lecture 1:
绪论\footnote{参考了计算机92班崔晨晨的笔记} \\ }
\textsl{Lecturer: 冯博琴 \hfill Scriber: 戴唯思}\\ \hline
\end{tabularx}
\setcounter{section}{0}
\renewcommand{\thepage}{\lecture -\arabic{page}}
\def\lecture{1}

\subsection{课程性质}

    原理性课程: 基础性, 科学性, 普适性, 针对性差

\subsection{学习的目的}

    \begin{itemize}
        \item 实现(通用)编译器
        \item 实现专用编译器
        \item 学习编译思想 \\
            计算思维: 抽象, 自动化
    \end{itemize}

\subsection{如何学}

    \begin{itemize}
        \item 把握三年级的特点
        \item 纪律: 准时到, 认真听, 注意记
        \item 注意学习方法
    \end{itemize}

\section{什么叫编译程序}

    \subsection{编译程序历史}

        \begin{itemize}
            \item 编译程序是系统软件中资格最老的成员之一
            \item 发展十分迅速和成熟
            \item 系统化的理论和技术
        \end{itemize}


    \subsection{编译理论与其他课程关系}

    \subsection{编译理论的应用}

    \subsection{编译程序和翻译程序}

        \textbf{翻译程序}: 输入是某种语言, 输出是另一种语言

    \subsection{编译程序}

        高级语言源程序通过编译程序生成面向机器的代码, 装配成目标程序代码.

\section{编译过程概述}

    \subsection{编译过程的组成}

        源程序通过\textbf{词法分析}分解为单词和符号, 通过\textbf{语法分析}被识别为语法单位, 通过\textbf{中间代码生成}产生中间代码, 通过\textbf{代码优化}和\textbf{目标代码生成}产生目标代码.
