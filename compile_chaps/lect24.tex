\def\lecture{24}
\clearpage \noindent\begin{tabularx}{\linewidth}{|X|}
\hline \vskip -2mm
{\sf 编译原理} \hfill November 25, 2011 \\
{\centering \sf \large Lecture \lecture:
运行时空间组织 \\ }
\textsl{Lecturer: 冯博琴 \hfill Scriber: 戴唯思}\\ \hline
\end{tabularx}
\setcounter{section}{0}
\renewcommand{\thepage}{\lecture -\arabic{page}}

\section{运行时空间组织}

    分配策略: 动态分配, 静态分配. 允许递归的语言不能使用静态分配. 满足先申请后释放的用栈实现, 否则用堆实现.

    \subsection{Fortran语言的存储分配}

        利用DAG实现分层分配. 局部区中包含临时变量, 形式单元, 寄存器保护区和返回地址. 特色: COMMON无名/有名公共数据块. 还有个很奇葩的等价语句, 可以采用并查集实现, 难点: 复杂变量, 方法是引入动态基准的坐标, 建立等价环.
