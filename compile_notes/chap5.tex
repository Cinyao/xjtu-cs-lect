\newcommand\eqdot{=\hskip -.8em\cdot\ }

\chapter{自下而上的语法分析}

    从输入串开始, 逐步进行\textsf{归约}, 直到到达文法的开始符号, 恢复一系列token表述的结构.

    \section{基本问题}

        在任何时刻, 找出句型的句柄.

        \subparagraph{归约} 将符号放入栈, 当栈顶形成某个表达式的一个候选式时将这部分归约为对应产生式的左部符号.

        \subparagraph{句柄} 从$uvw$到$uAw$的归约过程中, $A\to v$是一个句柄.

        \subparagraph{短语} 令$S$是文法$G$的开始符号, 若$\alpha\beta\delta$是$G$的一个句型, 若有$S\stackrel{*}{\Rightarrow}\alpha A\delta$且$A\stackrel{+}{\Rightarrow}\beta$, 则称$\beta$是句型$\alpha\beta\delta$相对于非终结符$A$的\textsf{短语}. 如果有$A\to\beta$, 则称$\beta$是句型$\alpha\beta\delta$相对于规则$A\to\beta$的\textsf{直接短语}. 一个句型的\textsf{最左直接短语}称为该句型的\textsf{句柄}. \textsf{规范推导}即最右推导. \textsf{规范句型}即由规范推导得到的句型.
        % 课件上这里写错了

        \subparagraph{规范归约} 关于句型$\alpha$的一个最右推导的逆过程, 也称为\textsf{最左归约}.

        \paragraph{在哪里寻找句柄}
            栈顶

        \paragraph{动作}
        \begin{description}
            \item[移进] 下一输入符号移进栈顶
            \item[归约] 把句柄按产生式的左部进行归约
            \item[接受] 分析程序报告成功
            \item[出错] 发现了一个语法错, 调用出错处理程序
        \end{description}

        被归约的串总是出现在栈顶.

    \section{直观算符优先分析法}

        任二个相继出现的终结符$a$与$b$(可能中间有$V_N$)的优先级可能有以下优先关系: $a\lessdot b$, $a\eqdot b$, $a\gtrdot b$. 采用\textbf{优先级表}\footnote{注: 我想到了中缀表达式转后缀表达式的算法\ldots}\footnote{明神指出, 严蔚敏《数据结构》里有这个例子, 一模一样}!
        
        直观算符优先文法仅仅按照优先级分析, 利用优先级的思想, 产生算符优先文法

    \section{算符优先分析}

        \subsection{算符优先文法}

            \begin{enumerate}
                \item 不存在$A\to \cdots NQ\cdots$
                \item 文法规定$V_T$符号被归约的顺序
                \item 两个$V_T$符号的优先关系至多存在一个
            \end{enumerate}

            关键: \textsf{优先表}(构造, 使用的算法)
            如果一个文法的任何产生式右部都不含两个相继(并列)的非终结符, 即不含有$\cdots QR\cdots$形式的产生式右部则我们称该文法为\textsf{算符文法}. 假定$G$是一个不含$\varepsilon-$产生式的算符文法. 对于任何一对终结符$a$, $b$, 我们说: 
            \begin{itemize}
                \item $a\eqdot b$当且仅当文法$G$中含有形如$P\to \cdots ab\cdots$或$P\to \cdots aQb\cdots$的产生式
                \item $a\lessdot b$当且仅当$G$中含有形如$P\to \cdots aR\cdots$的产生式, 而$R\stackrel{+}{\Rightarrow}b\cdots$或$R\stackrel{+}{\Rightarrow}Qb\cdots$
                \item $a\gtrdot b$当且仅当$G$中含有形如$P\to \cdots Rb\cdots$的产生式, 而$R\stackrel{+}{\Rightarrow}\cdots a$或$R\stackrel{+}{\Rightarrow}\cdots aQ$
            \end{itemize}
            如果一个算符文法$G$中的任何终结符对$(a,b)$至多只满足$a\eqdot b$, $a\lessdot b$, $a\gtrdot b$之一, 则称$G$是一个\textsf{算符优先文法}.

        \subsection{从算符优先文法构造优先关系表}

            \subparagraph{定义}

                % TODO

            \subparagraph{定理}

                算符优先文法句型的一般形式: $\#N_1a_1N_2a_2\cdots N_na_nN_{n+1}\#$, 其中$a_i\in V_T$, $N_i\in V_N|\wedge$.

                \textsf{最左素短语}是满足以下条件的最左子串$N_ja_j\cdots N_ia_iN_{i+1}$: $a_{j-1}\lessdot a_j$, $a_j\eqdot a_{j+1}, \cdots, a_{i-1}\eqdot a_i$, $a_i\gtrdot a_{i+1}$.

                % 判断时忽略$V_N$符号

                % 单非产生式

            \subparagraph{优缺点}

                速度快, 会误判

                \begin{table}[h!]
                    \centering
                    \caption{规范归约 VS 算符优先文法}
                    \label{tab:gfgyvcsfyx}
                    \begin{tabular}{ccc}\toprule
                        & 规范归约 & 算符优先文法 \\ \midrule
                        操作对象 & 句柄 & 最左素短语 \\
                        定义 & 最左直接短语 & --- \\
                        成分 & 文法符号 & 至少含有一个$V_T$符号 \\
                        位置 & 最左 & 最左 \\
                        怎么找 & 穷举(栈顶) & $\lessdot\cdots\eqdot\cdots\gtrdot$形式 \\
                        归约速度 & 慢 & 快 \\
                        原因 & --- & 忽略单非产生式 \\
                        正确性 & --- & 可能接受错误的输入 \\
                        \bottomrule
                    \end{tabular}
                \end{table}

        \subsection{一个优化: 引入优先函数}

            引入优先函数来代替优先表, 可以满足大多数情况的需求, 但无法表示语法上的错误. 实质是变相的查表. 优点: 节约存储空间; 缺点: 不能正确表示表格中``出错''(不可比较)的情况.

            不唯一.

            \subparagraph{存在性} 并非总能映射! 证明: 举反例.

            \subparagraph{生成方法} 有向图, 类似拓扑排序. $f(a)=$从$f_a$组开始的路径长度和.

    \section{LR分析法}

        LR分析: 从左到右扫描输入串, 识别句柄, 自下而上归约

        \paragraph{初态}

            $(S_0, \#, a_1,a_2, \cdots, a_n, \#)$

        \paragraph{一般情况}

            $(S_0S_1\cdots S_n, \#x_1x_2\cdots x_m, a_ia_{i+1}\cdots a_n\#)$

        \paragraph{动作}

            \begin{enumerate}
                \item ACTION$[S_m,a_i]$为移进, $S=GOTO[S_m,a_i]$:
                    \[(S_0S_1\cdots S_mS, \#x_1x_2\cdots x_ma_i, a_{i+1}\cdots a_n\#)\]
                \item ACTION$[S_m,a_i]=\{A\to B\}$, 则
                    \[(S_0S_1\cdots S_{m-1}S, \#x_1x_2\cdots x_{m-r}A, a_ia_{i+1}\cdots a_n\#)\]
                    $S=GOTO[S_{m-r},A]$, $r=|\beta|$, $\beta=x_{m-r+1}\cdots x_m$
                \item AC
                \item ERR
            \end{enumerate}

        \subsubsection{LR分析程序}

            \paragraph{实质} 分析栈+DFA

                
