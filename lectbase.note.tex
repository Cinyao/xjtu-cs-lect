\usepackage{amsthm,amssymb,fullpage,amsmath}
\usepackage[slantfont,boldfont,CJKnumber,CJKtextspaces,CJKmathspaces]{xeCJK}
\usepackage{tabularx}
\usepackage{graphicx}
\usepackage{booktabs}
\usepackage{multicol}
\usepackage{pstricks}
\usepackage{pst-node}
\usepackage{pst-blur}
\usepackage{rotating}
\usepackage[
    pdftitle={西安交通大学计算机系课堂笔记 http://code.google.com/p/xjtu-cs-lect/}
]{hyperref}
\setCJKmainfont{細明體}
\setCJKsansfont{微軟正黑體}

\makeatletter
\renewcommand\section{\@startsection{section}{1}{\z@}%
{-3.5ex \@plus -1ex \@minus -.2ex}%
{2.3ex \@plus.2ex}%
{\Large \sf}}
\renewcommand\subsection{\@startsection {subsection}{2}{\z@}%
{-3.25ex\@plus -1ex \@minus -.2ex}{1.5ex \@plus .2ex}%
{\normalfont \large \sf}}
\renewcommand\subsubsection{\@startsection {subsubsection}{3}{\z@ }{-3.25ex\@plus -1ex \@minus -.2ex}{1.5ex \@plus .2ex}{\normalfont \normalsize \sf }}
\renewcommand\chapter{
\if@openright \cleardoublepage \else \clearpage \fi \thispagestyle {plain}\global \@topnum \z@ \@afterindentfalse \secdef \@chapter \@schapter
}
\renewcommand\paragraph{
\@startsection {paragraph}{4}{\z@ }{3.25ex \@plus 1ex \@minus .2ex}{-1em}{\normalfont \normalsize \sf }
}
\renewcommand\subparagraph{
\@startsection {subparagraph}{5}{\parindent }{3.25ex \@plus 1ex \@minus .2ex}
{-1em}{\normalfont \normalsize \sf  }
}
\makeatother

\def\d{\mathrm{d}}
\def\e{\mathrm{e}}
\renewcommand{\tablename}{表} 
\renewcommand{\figurename}{图} 

\setlength\parindent{2em}


\newtheorem{theorem}{定理}[section]


\theoremstyle{definition}
\newtheorem{definition}{定义}[section]
%\newtheorem*{proof}{证明}
\newtheorem*{caution}{注意}
\newtheorem{example}{栗}[section]
\newtheorem{algorithm}{算法}[section]
\newtheorem*{sol}{解}
\newtheorem*{note}{注}
